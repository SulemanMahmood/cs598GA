\textbf{\textit{Small Model:}} The small model consists of two smart devices, a controlling application and a user. The devices are a fire alarm and a smart lock. The controlling application in this case is communicating directly with both devices and the user. The controlling application listens for fire alarm notifications. When a fire alarm is received, it sends a command to unlock the door and notifies the user that there is a fire alarm.   If the controlling application fails to open the door, it moves on to notifying the user. The communication model assumes unreliable channel and uses acknowledgments to check if communication was successful. However, the system only retires to send the message a bounded number of times. Our communication model allows the message and the acknowledgment to be lost independent of each other. If sender doesn't receive the acknowledgment then it assumes that the message did not get through and resends the message. The receiver sends the acknowledgment once and moves on with its logic. It doesn't need to know if the acknowledgment reached the sender. This opens up the possibility of the same message arriving twice on the receiver. In our model, the extra message would be treated exactly like the first message and it will trigger the same operations. We rely on the fact that most IOT applications have a trigger action pattern and do not have side effects. i.e. executing the operations twice will not have any undesired effect on the state of the system. For the model the property being checked states that if there is a fire then doors should eventually be open and user must also be notified. In order to check this property, PRISM prepares a model with 567 states and 1526 transitions. We will use this model to to evaluate all three \textit{RQs}. 

\textbf{\textit{Large Model:}} A complete home security system is modeled for this example. The model is heavily based of commercially available system by SimpliSafe\cite{simplisafe}. The model includes a total of 17 components. We model 7 sensors, 4 devices, a hub, a cloud platform, a user, a burglar and fire and police responses. We took inspiration from numerous home security systems available in the market. The sensors include: entry sensor, motion sensor, glass break sensor, freeze detection sensor, flood detection sensor, smoke and carbon monoxide sensor. The additional devices are a keypad to disarm the house on entry, a keychain to arm the house, a panic button and a loud siren. All the sensors and devices connect to the hub which forwards all the notifications to the cloud platform which in turn notifies user and alert the proper authorities if necessary. This model is quite large therefore difficult to have a complete unreliable communication model for this. Therefore we have adopted a simpler communication model. The probability of successful message delivery is calculated on the basis of number of retries and probability of successful delivery of a single message. If the probability of successful delivery of single message is \textit{p}, then the probability of successful delivery in \textit{n} retries is $ 1 - (1 - p)^{n} $. This communication model doesn't capture the complexities of retransmission that were captured in the smaller model. Since re-transmission doesn't add any time delay in this communication model, we will not use this model for \textit{RQ3}. It is also possible for the user to physically interact with the environment which will trigger a series of actions. The messages that represent physical interaction are not probabilistic. e.g User presses the panic button.

The property that we verify starts with the condition that there is a safety warning in the house. This warning can be any one of the 5 situations: smoke, carbon monoxide, freezing temperatures, flooding and burglar. At the end we want the situation to be cleared. This may seem like a departure from our definition using data inconsistencies but this definition is equivalent because once the user or the right authorities have information about the situation, the modeling assumes that they will certainly act to clear it up. The situation with the burglar is unique because a burglar will trigger each sensor with some probability. Therefore it is possible that a burglar doesn't trip any sensors and goes unnoticed. A simulation for message success probability of 1 using 100000 samples and path length of 100 shows that burglar will be caught with a probability 98.86\% +/- 0.08\% at a confidence level of 99\%. This shows that the system is quite reliable for detecting burglars and calling the authorities in a short duration when networks are reliable.

\begin{table}[]
\centering
\caption{Characteristics of Large Model}
\label{table1}
\begin{tabular}{|l|l|l|l|}
\hline
Property Description                        & States     & Transitions & Approx Verification Time \\ \hline
Flooding. Message Success Probability = 0.9 & 40176      & 251910      & 80s                      \\ \hline
Burglar. Message Success Probability = 1    & 612540416  & 7371907072  & Out of Memory Error      \\ \hline
Burglar. Message Success Probability = 0.9  & 5438223360 & 74182033152 & Out of Memory Error      \\ \hline
\end{tabular}
\end{table}

The full model is significantly more complex than the small model we saw earlier. Table \ref{table1} show a summary of the number of states and transitions for the models generated for different properties. Some simple alarms, like flood, the model is still feasible for verification through analytical means. When we try to work with the burglar alarm the model quickly explodes and becomes infeasible for analytical verification. The state explosion would have been worse if full communication model had been used. For this reason, we will use simulation for verifying properties on this model. One limitation of prism is that it cannot automatically chose between a number of possible initial state when it is using simulation. For this reason we have to specify which alarm to trigger. We selected burglar alarm since it is the hardest to verify analytically.