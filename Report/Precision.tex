\subsection{Precision Reduction Benchmarks}
The programs for LU Decomposition and Matrix inversion use double precision floating point numbers. The LU Decomposition benchmark solves a set of 6 linear equations and the Matrix inversion benchmark calculates inverse of a 3 by 3 matrix. In both cases use of double precision numbers open up the opportunity for reducing the precision to single precision floating point numbers for calculation purposes. This approximation works only if small numbers are not involved in division operations.

The accuracy measure for the LU decomposition uses the solution of the equations. The problem is set up such that the solution is always a vector of all ones. This is achieved by setting the b vector to be the sum of all entries in a row. Since we know that the idea solution is a vector of all ones we can measure deviation from this vector as a measure of error. The difference vector is calculated by subtracting the vector of all ones from the single precision solution vector. The ratio of the magnitude of the difference vector and the magnitude of all ones vector is a measure of the relative error. We estimate accuracy by (1 - relative error) for LU decomposition. The accuracy measure for matrix inversion is defined in a similar manner. The difference between the inverse calculated using double precision and single precision numbers is calculated. This difference matrix is treated as a 9 dimensional vector and its magnitude is divided by the magnitude of the double precision matrix (treated as 9-D vector). This ratio gives us relative error. As with LU decomposition, accuracy is defined as (1 - relative error).

For both benchmarks, A set of 1000 randomly initialized matrices is created. For LU decomposition the matrices are 6 by 6. For Matrix inversion the matrices are 3 by 3. We calculate accuracy for each of these matrices. For WCET estimate for LU decomposition dropped to 54.87\% of exact calculation. The WCET estimate for matrix inversion dropped to 54.26\% of the exact calculation. On the other hand the accuracy for both programs was 99.9999\% with standard deviation of 0.0004\%. The answers to our common questions are as follows:

\begin{enumerate}
\item There is no knob here but turning on the approximation showed negligible degradation in the results.
\item WCET dropped to 54\% of the original value. This seems very promising.
\item From the results presented here. It seems that 46\% of the work is done to get the last 0.0001\% of the accuracy. 
\end{enumerate}

As the last point above suggests there is very little gain for a large amount of work but it must be remembered that the data we are working with can have a huge impact on the results. In particular if the matrices being used have very wide rage of magnitude differences then double precision variables are important for accuracy. The take away message from this benchmark is that understanding the condition under which a system is going to work can help us in selecting the appropriate precision level for the program.
