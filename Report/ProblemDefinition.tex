\section{Experiment Subjects}
The first step in performing the experiments was the selection of programs that can be used for approximations and WCET measurements. TACLe Benchmarks \cite{Taclebench} is a benchmark suite for WCET estimation tools. This benchmark combines multiple benchmarks for real time systems. The benchmark contains 53 programs which are subdivided into 5 categories. The kernels category includes 23 programs. All 23 programs from the kernel category were manually analyzed to assess approximation feasibility. 8 programs were found to have features that could be used for approximation. Table \ref{description} shows the 8 benchmarks, the description of the program and the approximation used in the program.

One of the major advantages of using TACLe Benchmarks was the presence of annotations that WCET tools expect from the developer. The paper for TACLe Benchmarks showed that the benchmark was checked for errors using a PATMOS \cite{Patmos} simulator. The PATMOS simulator is part of the T-Crest project \cite{tcrest} which aims to design computer architectures that have predictable execution times for the program. They are trying to address the second of the two challenges that we identified for WCET analysis. The problem is to make processors more time predictable while keeping the advances of last 4 decades in place. The project has proposed a new architecture called PATMOS and their Github repository \cite{gtcrest}  provides numerous tools related to this architecture. In this project, their llvm and clang port has been used for compiling the subjects. The PATMOS simulator, PASIM, has been used to run the programs and static analysis based WCET tool, PLATIN, has been used for WCET estimates. More accurate WCET estimates can be achieved by using Absint's aiT tool but it is not free.

\begin{table}[]
  \centering
  \caption{Benchmarks and their descriptions}
  \label{description}
  \begin{tabular}{|l|l|l|}
    \hline
    \textbf{Program} & \textbf{Description}                   & \textbf{Approximation}           \\ \hline
    BinarySearch     & Search in a sorted array               & Input Reduction                  \\ \hline
    BubbleSort       & Sort an unsorted array                 & Loop Perforation                 \\ \hline
    CountNegative    & Count negative numbers in a matrix     & Under-sampling and Extrapolation \\ \hline
    LUDecomposition  & Performs LU decomposition of a matrix  & Precision Reduction              \\ \hline
    MatrixInverse    & Calculate inverse of 3 X 3 matrix      & Precision Reduction              \\ \hline
    Prime            & Checks is a number is prime            & Precision Reduction              \\ \hline
    QuickSort        & Sort an unsorted array                 & Recursion Limiting               \\ \hline
    Stats            & Compute statistical measures on arrays & Loop Truncation                  \\ \hline
  \end{tabular}
\end{table}

All the experiments presented in this report were conducted on the default processor configuration. The default configuration is common for the clang port, simulator and static WCET tool. Some features of the processor configuration are listed below:
\begin{itemize}
\item Number of Cores = 1
\item Processor frequency = 80MHz
\item Memory attached = 2MB
\item Memory transfer time = 21 cycles
\item Transfer burst size = 16 bytes
\item Data cache = 2kB Direct Mapped Cache with line size of 128bits
\item Instruction cache = 4kB FIFO Cache with line size of 128bits
\end{itemize}

In order to calculate the accuracy the exact and approximate version of the program are combined into a single program. This program is compiled using gcc and executed to obtain accuracy measurements.

%% Next section on results
