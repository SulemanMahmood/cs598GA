\section{Introduction}
Real time systems impose timing constraints on computation. Consider a
computer controlled manufacturing process as an example. Multiple
machines may be moving and working on the same object. A machine
moving towards an object needs to know that the current process on the
object will be finished by the time it reaches the object to start its
work. This imposes a strict timing constraint on the current
process. Since the machines are controlled by computers these timing
constraints translate directly to timing constraints on the
computations being done by the computer. A violation of these
constraints may result in machines colliding with each other resulting
in damage to both machines. Such systems, where even a single
violation of a timing constraint can result in damage, are called hard
real time system. The requirements of a real time system can be
divided into two parts. 1) Program must finish within the required
time limit. 2) Programs must produce the correct output. Our example
with interacting machines clearly shows that there is no room for
compromise in the first requirement. But the requirement about correct
result can be relaxed. The results produced by the computations should
be good enough for the system to work. For example, computing a value
up to 15 significant digits is pointless when the actuators have a
movement precision of 4 significant figures. This project presents
multiple benchmarks and their worst case running time with various
approximations. The results show that many approximations in the
programs retain reasonable accuracy in the results while producing
significant reduction in expected worst case timing.
