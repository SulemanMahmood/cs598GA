\section{Introduction}
In recent years, home automation through smart devices has gained popularity. These devices support remote monitoring and control of the devices through interfaces exposed to the networks. These networks can be local networks, allowing monitoring and control when the user is connected to the same network, or a connection to cloud services over Internet allowing monitoring and control from anywhere in the world.

The connection with cloud opens up a world of possibilities. Cloud services like IFTTT and Amazon Web-services for (Internet of Things) IOT allow users to specify rules to control their devices and perform actions without active user interaction. These rules can also use data from external services to help with the decisions. For example, a user can specify the rule that garage lights on his house should turn on everyday at sunset and the lights should turn off at sunrise. The time for sunrise and sunset vary everyday therefore the cloud services triggering this action will take input from an external service to determine the time of sunset and sunset at the location of the house and trigger actions accordingly. These rules can also be used to create dependencies between devices where the value read from one device is used to trigger an action on another device. For example, a user may program a rule that when there is a fire alarm then all locks should open to let people escape.

The rules enforced by the cloud services and actions triggered by the user are classic examples of event driven systems. The conventional thinking about event driven systems says that every event in equally important and necessary for correct operation of the system. Recent results in the domain of approximate computing challenge this notion. In some of our experiments with numerical programs with multiple threads of execution and synchronization, a significant number of messages between different threads were dropped. This produces an wrong result. If the difference between the wrong result and correct result is small enough to be ignored, then we say that the result is approximately correct with acceptable error. We discuss these experiments in detail in a later sections. The key conclusion from those experiments is that acceptable results can be achieved from a system even when all messages are not delivered.

In this project, our hypothesis is that an event driven system of connected devices can also achieve acceptable results when message delivery is not guaranteed. The core idea behind the hypothesis is that a number of events flowing through the system are just regular updates about the status of the devices. Even when some of the updates are dropped the system should continue working properly as long as some updates reach their destination. For example, consider a temperature sensor sending temperature to the thermostat every second. Now assume that 90\% of the messages from the temperature sensor are dropped. This would result in the thermostat having outdated temperature values but on average the temperature measurement is only 10 seconds outdated. This results in thermostat working with an average of 10 second delay which should be acceptable in common home systems. Current IOT systems are built on the assumption that every message is equally important therefore devices ensure that their messages are delivered through energy expensive re-transmissions and acknowledgments. If we can show that dropping messages sent by a device will have a negligible effect on the whole system, then the communication constraints, ensuring message delivery, on that device can be reduced. This is particularly important for battery powered devices because communication is often the most energy intensive operation. Reducing this energy consumption will result in longer battery life. 

Our approach to test our hypothesis is based on statistical model checking. An IOT system is modeled using Discrete Time Markov Chain (DTMC) in well known model checking tool PRISM. We define a set of properties that capture situations that we believe, represent unacceptable system behavior. If the model has race conditions or other intrinsic problems then system can reach an unacceptable state even when communication is reliable. We rue out this possibility by checking our system models with 100\% reliable communication. Given 100\% reliable communication our model should show that unacceptable states are not possible. Once the system is verified with reliable communication, we introduce message dropping in the model. In order to evaluate our hypothesis, we ask the question: How likely is the unacceptable behavior with unreliable communication? We study the variation of this probability under different message delivery probabilities which represent a characteristic of the channel. If the likelihood of unacceptable state in unreliable channel is too high then bounded retries can be introduced to reduce some of the risk. We also evaluate the effect of a bounded number of retries to deliver a message on the likelihood of reaching an unacceptable state. We have performed this analysis on a simple model involving four components. In order to evaluate the scalability of the approach, we have also modeled a larger IOT system consisting of 20 components. Our results show that even small number of retries can result in a system that has very low chance of reaching an unacceptable state.
